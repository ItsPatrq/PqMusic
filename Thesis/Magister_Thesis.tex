\documentclass[12pt,a4paper,twoside]{mwart}
\usepackage[T1]{fontenc}
\usepackage[utf8]{inputenc}
\usepackage{polski} \usepackage{indentfirst}
\usepackage[left=3cm,right=2cm,top=2.5cm,bottom=2.5cm]{geometry}
\usepackage{todonotes}
\usepackage{listings}
\usepackage{float}
\usepackage{hyperref}
\usepackage{dirtree}
\usepackage{fancyhdr}
\usepackage[nottoc,numbib]{tocbibind}
\usepackage{apacite}
\usepackage{enumitem}

\setlength{\parskip}{13pt plus 2pt minus 3pt} %% Odstęp dla paragrafów
\linespread{1.15}

\newcommand{\makecell}[2][@{}c@{}]{\begin{tabular}{#1}#2\end{tabular}}
\newcommand{\file}[1]{\textbf{\textit{\textcolor{darkgreen}{#1}}}}
\newcommand{\folder}[1]{\textbf{\textit{\textcolor{dodgerblue}{#1}}}}

\pagestyle{fancy}
%% JN: nie jestem pewien czy klasa 'article' z 'openany' ma \markright
%% JN: uproszczenie paginy: tylko autor i tytuł pracy, zamiast tytułu
%%     bieżącego rozdziału.
\renewcommand{\sectionmark}[1]{\markright{#1}}
\fancyhf{}
\fancyhead[LO]{\small \emph{\nouppercase{\leftmark}}}
\fancyhead[RE]{\small \emph{\nouppercase{\rightmark}}}

\cfoot{\fancyplain{}{\thepage}}

% Czesc odpowiedzialna za styl kodu
% Taken from Lena Herrmann at 
% http://lenaherrmann.net/2010/05/20/javascript-syntax-highlighting-in-the-latex-listings-package
\usepackage{listings}
\usepackage{color}
\definecolor{lightgray}{rgb}{.9,.9,.9}
\definecolor{darkgray}{rgb}{.4,.4,.4}
\definecolor{purple}{rgb}{0.65, 0.12, 0.82}
\definecolor{dodgerblue}{RGB}{28, 134, 238}
\definecolor{darkgreen}{RGB}{0,153,51}
\definecolor{darkOrange}{RGB}{120,30,0}

\lstdefinelanguage{JavaScript}{
  keywords={typeof, new, true, false, catch, function, return, null, try, catch, switch, var, let, if, in, while, do, else, case, break},
  keywordstyle=\color{purple}\bfseries,
  ndkeywords={class, export, boolean, throw, implements, import, this, Math, Float32Array, Array},
  ndkeywordstyle=\color{blue}\bfseries,
  identifierstyle=\color{black},
  sensitive=false,
  comment=[l]{//},
  morecomment=[s]{/*}{*/},
  commentstyle=\color{darkgreen}\ttfamily,
  stringstyle=\color{darkOrange}\ttfamily,
  morestring=[b]',
  morestring=[b]"
}

\lstset{
   language=JavaScript,
   backgroundcolor=\color{lightgray},
   extendedchars=true,
   basicstyle=\footnotesize\ttfamily,
   showstringspaces=false,
   showspaces=false,
   %% JN: jeśli ie odwołuje się do konkretnych linii, to ich numerowanie nie do końca ma sens
   %% JN: jesli numerowanie linii jest uzywane do pokazania gdzie linie zostały zawiniete,
   %%     to lepiej uzyć mniej agresywnego stylu
   numbers=left,
   numberstyle=\footnotesize,
   numbersep=9pt,
   tabsize=2,
   breaklines=true,
   showtabs=false,
   captionpos=b
}

\setlength {\marginparwidth }{2cm}
\begin{document}
  
\begin{titlepage}
	\begin{center}
		\large Uniwersytet Mikołaja Kopernika w Toruniu\\
		\large Wydział Matematyki i Informatyki\\
		\vspace{3cm} 
		\large Patryk Bieszke\\
			nr albumu: 273187\\
			informatyka\\
		\vspace{2cm}
		Praca magisterska\\
	
		\vspace{3cm} 
		\huge Generowanie kompozycji muzycznych z użyciem sieci neuronowych\\
	\end{center}
	\hfill
	\begin{minipage}{6cm}
		\vspace{3cm}
		Opiekun pracy dyplomowej\\
		prof. dr hab. Krzysztof Stencel
	\end{minipage}
	\vspace{4cm}
	\begin{center}
		Toruń 2018\\
	\end{center}
\end{titlepage}


\pagenumbering{gobble}% Remove page numbers (and reset to 1)

\bibliographystyle{apacite}
\bibliography{References}
\clearpage
\thispagestyle{empty}
\mbox{}

\pagenumbering{arabic}
\tableofcontents 

\clearpage

\setcounter{secnumdepth}{0}
\section{Wstęp}
\label{sec:wstep}

\newpage

\section{Słownik skrótów}
\begin{itemize}
\item \textbf{DAW} - \textit{ang. \textbf{D}igital \textbf{A}udio \textbf{W}orkstation}, pol. Cyfrowa Stacja Robocza 
\item \textbf{MIDI} - \textit{ang. \textbf{M}usical \textbf{I}nstrument \textbf{D}igital \textbf{I}nterface}
\item \textbf{AUX} - \textit{ang. \textbf{Aux}iliary}
\item \textbf{W3C} - \textit{ang. \textbf{W}orld \textbf{W}ide \textbf{W}eb \textbf{C}onsortium}
\item \textbf{SVG} - \textit{ang. \textbf{S}calable \textbf{V}ector \textbf{G}raphics}
\item \textbf{HTML} - \textit{ang. \textbf{H}yper\textbf{t}ext \textbf{M}arkup \textbf{L}anguage}
\item \textbf{CSS} - \textit{ang.  \textbf{C}ascading \textbf{S}tyle \textbf{S}heets}
\item \textbf{JS} - \textit{\textbf{J}ava\textbf{S}cript}
\item \textbf{DOM} - \textit{ang. \textbf{D}ocument \textbf{O}bject \textbf{M}odel}
\item \textbf{API} - \textit{ang. \textbf{A}pplication \textbf{P}rogramming \textbf{I}nterface}
\item \textbf{CLI} - \textit{ang. \textbf{C}ommand \textbf{L}ine \textbf{I}nterface}
\item \textbf{CQRS} - \textit{ang. \textbf{C}ommand \textbf{Q}uery \textbf{R}esponsibility \textbf{S}egregation}
\item \textbf{MVC} - \textit{ang. \textbf{M}odel-\textbf{V}iew-\textbf{C}ontroller}
\item \textbf{URL} - \textit{(ang. \textbf{U}niform \textbf{R}esource \textbf{L}ocator )}
\item \textbf{npm} - \textit{(ang. \textbf{N}ode \textbf{P}ackage \textbf{M}anager)}
  \todo[inline]{Czy jest to dobra konwencja dla słownika skrótów? Czy umieścić go przed wstępem z racji tego, że we wstępie chciałbym użyć DAW?}
  
\end{itemize}
\newpage
\setcounter{secnumdepth}{2}

\section{Transkrypcja muzyki}
Pliki dźwiękowe przechowują sygnały dźwiękowe w postaci sugnałów cyfrowych. Układ w którym pliki te przechowują bity jest zdeterminowany przez kodowanie, które zostało na nich zastosowane. Istnieje wiele kodowań przeznaczonych do danych dźwiękowych, zarówno nieskompresowanych jak i skompresowanych. Te drugie powtały z myślą o zaoszczędzeniu jak największej ilości pamięci w zaleności od tego, w jakim stopniu mona pójść na kompromis z jakością skompresowanego sygnału.

Z punktu widzenina algorytmów generujących muzkę, w tym modeli sieci, pliki dźwiękowe nie są wystarczająco funkcjonalną formą trzymania danych. W celu pozyskania istotnych informacji w kontekście maszynowej kompozycji muzyki nalezałoby zastosować szereg algorytmów, w tym transformacji i funkcji statystycznych, na każdym z plików. Proces ten jest skomplikowany pod względem obliczeniowym i wymaga nie małej mocy obliczeniowej. Z racji tego, że kategoryzacje zbioru muzycznego jak i dynamiczne dostosowywanie modelu, na podstawie którego zostaną generowane kompozycje muzyczne według założeń projektu ma odbywać się w czasie rzeczywistym, jedyną formą przechowywania plików z informacją o kompozycjach będzie postać po zastosowaniu transkrypcji na danych utworach.

%% z PWN : transkrypcja [łac. transcriptio ‘przepisywanie’], muz. opracowanie utworu muz. na inny niż w oryginale zespół wykonawczy (orkiestracja) lub inny instrument; Pytanie czy rozszerzenie tego znaczenia w języku polskim jest poprawne? także zapisanie utworu inną notacją muz.; transkrybować — zapisywać znakami współcz. notacji muz. utwory zapisane dawnymi systemami notacyjnymi, np. notacją modalną, menzuralną. Po ENG transkrypcja jest zgodna z rozdziałem

Transkrypcja muzyki to proces polegający na zapisaniu danego utworu muzycznego w sposób formalny, taki jak zapis nutowy, z istniejącego zapisu dźwiękowego. Operacja ta wykorzystywana jest do lepszego zrozumienia muzyki, która wcześniej nie była zapisana w taki sposób Jest to skomplikowany i czasochłonny proces nawet dla profesjonalistów z tej dziedziny. Algorytmiczne podejście do tego problemu obejmuje rozpoznanie rytmu i wysokości tonu poprzez analize sygnału fonicznego. Do polepszenia rezultatu tego procesu często wymagane jest odesparowanie od siebie poszczególnych instrumentów w danym utworze muzycznym, ponieważ nakładanie się na siebie sygnały odrębnych źródeł stanowi jeden z większych problemów transkrypcji

W tym rozdziale przedstawię wciąż nierozwiązany problem automatycznej transkrypcji muzycznej, nakreślając najbardziej problematyczne rejony na przykładach istniejących rozwiazań.  Celem tych rozważań jest opisanie modułu opisywanego projektu, służącego do transpilacji wysłanego przez użytkownika poliku dźwiękowego przez serwer

\subsection{Jak "działa" muzyka}

\subsection{Transformata Fouriera}

\subsection{Estymacja składowej fundamentalnej}

\subsection{Estymacja wielotonowa}


\newpage
\section{Klasyfikacja muzyki}

\section{Generowanie muzyki}

\section{Implementacja}

\section{Zakonczenie}

OKok
\end{document}  
